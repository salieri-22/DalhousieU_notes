\documentclass[11pt]{article}

\usepackage{amsmath}
\usepackage{amssymb}
\usepackage{amsthm}
\usepackage{hyperref}
\usepackage{microtype}
\usepackage{graphicx}
\graphicspath{ {./img/} }

\setlength{\parindent}{0cm}
\let\emptyset\varnothing

\title{\textbf{MATH 2135 Linear Algebra} \\ 3.C Matrices}
\author{Alyssa Motas}

\begin{document}

    \maketitle

    \pagebreak

    \tableofcontents

    \pagebreak

    \section{Representing a Linear Map by a Matrix}

    \subsection{Definition of a matrix}

    Let $m$ and $n$ denote positive integers. An $m$-by-$n$ \emph{matrix} is a rectangular array of elements of \textbf{F} with $m$ rows and $n$ columns:
    \begin{equation*}
        A = \begin{bmatrix}
                A_{1,1} & \dots & A_{1,n} \\
                \vdots  &       &  \vdots \\
                A_{m,1} & \dots &  A_{m,n}
            \end{bmatrix}
    \end{equation*}
    The notation \(A_{j,k}\) denotes the entry in row $j$, column $k$ of $A$.

    \subsection{Definition of the matrix of a linear map}

    Suppose \(T \in \mathcal{L}(V,W)\) and \(v_1, \dots, v_n\) is a basis of $V$ and \(w_1, \dots, w_m\) is a basis of $W$. The \emph{matrix} of $T$ with respect to these bases is the $m$-by-$n$ matrix \(\mathcal{M}(T)\) whose entries \(A_{j,k}\) are defined by \[Tv_k = A_{1,k} w_1 + \dots + A_{m,k} w_m.\] If the bases are not clear from the context, then the notation \(\mathcal{M}(T,(v_1, \dots, v_n),(w_1, \dots, w_n))\) is used.
    \begin{equation*}
        \mathcal{M}(T) = A = \begin{bmatrix}
                                a_{1,1} & a_{1,2} & \dots & a_{1,n} \\
                                \vdots  &  \vdots     &       &   \vdots \\
                                a_{m,1} & a_{m,2} & \dots & a_{m,n}
                             \end{bmatrix}
    \end{equation*}

    \subsection{Example}
    Suppose \(T \in \mathcal{L}(\textbf{F}^2, \textbf{F}^3)\) is defined by \(T(x,y) = (x+3y, 2x+5y, 7x + 9y).\) Find the matrix of $T$ with respect to the standard bases of \(\textbf{F}^2\) and \(\textbf{F}^3\). 

    \begin{proof}[\unskip\nopunct]
        Since \(T(1,0) = (1,2,7)\) and \(T(0,1) = (3,5,9)\), then
        \begin{equation*}
            \mathcal{M}(T) = \begin{bmatrix}
                               1 & 3 \\
                               2 & 5 \\
                               7 & 9
                             \end{bmatrix}.
        \end{equation*}
    \end{proof}
    Suppose \(D \in \mathcal{L}(\mathcal{P}_3 (\mathbb{R}), \mathcal{P}_2 (\mathbb{R}))\) is the differentiation map defined by \(Dp = p'\). Find the matrix of $D$ with respect to the standard bases of \(\mathcal{P}_3 (\mathbb{R})\) and \(\mathcal{P}_2 (\mathbb{R})\).

    \begin{proof}[\unskip\nopunct]
        Since \((x^n)' = nx^{n-1}\), then we have
        \begin{equation*}
            \mathcal{M}(D) = \begin{bmatrix}
                                0 & 1 & 0 & 0 \\
                                0 & 0 & 2 & 0 \\
                                0 & 0 & 0 & 3
                             \end{bmatrix}
        \end{equation*}
    \end{proof}

    \section{Addition and Scalar Multiplication of Matrices}

    \subsection{Definition of matrix addition}

    The \emph{sum of two matrices of the same size} is the matrix obtained by adding corresponding entries in the matrices:
    \begin{equation*}
        \begin{bmatrix}
            A_{1,1} & \dots & A_{1,n} \\
            \vdots  &       & \vdots  \\
            A_{m,1} & \dots & A_{m,n}
        \end{bmatrix} + \begin{bmatrix}
                            C_{1,1} & \dots & C_{1,n} \\
                            \vdots  &       &  \vdots \\
                            C_{m,1} & \dots & C_{m,n}
                        \end{bmatrix} = \begin{bmatrix}
                                            A_{1,1} + C_{1,1} & \dots & A_{1,n} + C_{1,n} \\
                                            \vdots            &       &   \vdots          \\
                                            A_{m,1} + C_{m,1} & \dots & A_{m,n} + C_{m,n}
                                        \end{bmatrix}
    \end{equation*}
    In other words, \((A+C)_{j,k} = A_{j,k} + C_{j,k}.\)

    \subsection{The matrix of the sum of linear maps}

    Suppose \(S,T \in \mathcal{L}(V,W)\). Then \(\mathcal{M}(S + T) = \mathcal{M}(S) + \mathcal{M}(T)\). 

    \subsection{Definition of scalar multiplication of a matrix}

    The product of a scalar and a matrix is the matrix obtained by mutliplying each entry in the matrix by the scalar:
    \begin{equation*}
        \lambda \begin{bmatrix}
                    A_{1,1} & \dots & A_{1,n} \\
                    \vdots  &       &  \vdots \\
                    A_{m,1} & \dots & A_{m,n}
                \end{bmatrix} = \begin{bmatrix}
                                    \lambda A_{1,1} & \dots & \lambda A_{1,n} \\
                                    \vdots          &       & \vdots \\
                                    \lambda A_{m,1} & \dots & \lambda A_{m,n}
                                \end{bmatrix}
    \end{equation*}
    In other words, \((\lambda A)_{j,k} = \lambda A_{j,k}\)

    \subsection{The matrix of a scalar times a linear map}

    Suppose \(\lambda \in \textbf{F}\) and \(T \in \mathcal{L}(V,W)\). Then \(\mathcal{M}(\lambda T) = \lambda \mathcal{M}(T)\). 

    \subsection{Notation of \(\textbf{F}^{m,n}\)}

    For $m$ and $n$ positive integers, the set of all $m$-by$n$ matrices with entries in \textbf{F} is denoted by \(\textbf{F}^{m,n}\).

    \subsection{\(\dim \textbf{F}^{m,n} = mn\)}

    Suppose $m$ and $n$ are positive integers. With addition and scalar multiplication defined as above, \(\textbf{F}^{m,n}\) is a vector space with dimension $mn$. 

    \section{Matrix Multiplication}

    \subsection{Definition of matrix multiplication}

    Suppose $A$ is an $m$-by-$n$ matrix and $C$ is an $n$-by-$p$ matrix. Then $AC$ is defined to be the $m$-by-$p$ matrix whose entry in row $j$, column $k$, is given by the following equation: \[(AC)_{j,k} = \sum_{r=1}^{n} A_{j,r} C_{r,k}.\] In other words, the entry in row $j$, column $k$, of $AC$ is computed by taking row $j$ of $A$ and column $k$ of $C$, multiplying together corresponding entries, and then summing. Matrix multiplication is not commutative, but it is associative and distributive. 

    \subsection{The matrix of the product of linear maps}

    If \(T \in \mathcal{L}(U,V)\) and \(S \in \mathcal{L}(V,W)\), then \(\mathcal{M}(ST) = \mathcal{M}(S) \mathcal{M}(T)\). 

    \section{Isomorphism}

    \subsection{Definition of isomorphism}

    Let $V,W$ be vector spaces over \textbf{F} and let \(T \in \mathcal{L}(V,W)\). We say that $T$ is an \emph{isomorphism} if $T$ is bijective (i.e. injective and surjective. 

    \subsection{Inverse function}

    If \(T \in \mathcal{L}(V,W)\) is an isomorphism, then the inverse function \(T^{-1} \in \mathcal{L}(W,V)\) exists and is linear.

    \vspace{1em}

    \emph{Example.} Let \(V = \mathbb{R}^4\) and let \(W = \mathcal{P}_3 (\mathbb{R})\). A basis of $V$ is \(v_1 = (1,0,0,0), v_2 = (0,1,0,0), v_3 = (0,0,1,0), v_4 = (0,0,0,1)\), and a basis of \(\mathcal{P}_3 (\mathbb{R})\) is \(w_1 = 1, w_2 = x, w_3 = x^2, w_4 = x^3\). Define an isomorphism \(T \in \mathcal{L}(\mathbb{R}^4 \rightarrow \mathcal{P}_3 (\mathbb{R}))\), namely the unique linear map such that \[T(v_1) = w_1, \dots, T(v_4) = w_4.\] The inverse \(T^{-1} \in \mathcal{L}(\mathcal{P}_3 (\mathbb{R}) \rightarrow \mathbb{R}^4)\) is the unique linear map such that \[T^{-1} w_1 = v_1, \dots, T^{-1} w_4 = v_4.\] More concretely, we can describe $T$ and $T^{-1}$ like this:
    \begin{equation*}
        T \begin{bmatrix}
            a \\
            b \\
            c \\
            d  
          \end{bmatrix} = a + bx + cx^2 + dx^3
    \end{equation*}
    and
    \begin{equation*}
        T^{-1} (a + bx + cx^2 + dx^3) = \begin{bmatrix}
                                            a \\
                                            b \\
                                            c \\
                                            d
                                        \end{bmatrix}
    \end{equation*}

    \subsection{Definition of isomorphic}
    Vector spaces $V,W$ are \emph{isomorphic} if there exists an isomorphism between them.

    \subsection{Finite-dimensional vector spaces are isomorphic}
    
    Finite-dimensional vector spaces are isomorphic if and only if they have the same dimension. 

\end{document}