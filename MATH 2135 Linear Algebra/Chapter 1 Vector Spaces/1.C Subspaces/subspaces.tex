\documentclass[11pt]{article}

\usepackage{amsmath}
\usepackage{amssymb}
\usepackage{amsthm}
\usepackage{hyperref}
\usepackage{microtype}

\setlength{\parindent}{0cm}
\let\emptyset\varnothing

\title{\textbf{MATH 2135 Linear Algebra} \\ 1.C Subspaces}
\author{Alyssa Motas}

\begin{document}

    \maketitle

    \pagebreak

    \tableofcontents

    \pagebreak

    \section{Definition}

    Let $V$ be a vector space over a field $F$. A subset $U$ of $B$ is called a \emph{subspace} of $V$ if $U$ is also a vector space in its own right, using the same zero, addition, and scalar multiplication as $V$.

    \subsection{Characterization of Subspaces}

    A subset $U \subseteq V$ is a subspace if and only if $U$ satisfies the following three conditions:

    \begin{enumerate}
        \item[(1)] Additive identity. \[0 \in U\]
        \item[(2)] Closed under addition. \[\forall v,w, v,w \in U \Rightarrow v + w \in U\] 
        \item[(3)] Closed under scalar multiplication. \[\forall a,v, a \in F, v \ in U \Rightarrow av \in U\] 
    \end{enumerate}
    \begin{proof}
        ``\(\Rightarrow\)'' Given \(U \subseteq V\), assume $U$ is a subspace of $V$. We want to show that $U$ satisfies all three conditions above.
        \begin{enumerate}
            \item[(1)] By definition of subspaces, the zero vector of $V$ is the zero vector of $U$. So \(0 \in U\).
            \item[(2)] Since $U$ is a vector space, the sume of two vectors in $U$ is a vector in $U$. Also, $U$ uses the same addition operation as $V$. So whenever \(v,w \in U\), then \(v + w \in U.\)
            \item[(3)] Similar to (2).  
        \end{enumerate}
    \end{proof}

    \begin{proof}
        ``\(\Leftarrow\)'' Another proof is this: To show that $U$ is a vector space, we first need an element \(0 \in U\) and operations \[+ : U \times U \rightarrow U \qquad \text{and} \qquad \cdot : F \times U \rightarrow U.\] Second, we must show axioms (A1) - (M4).
        \begin{enumerate}
            \item[(1)] By assumption, \(0 \in U\), where 0 is the additive identity of $V$. So we can use 0 as the additive identity of $U$.
            \item[(2)] By assumption, $U$ is closed under addition, so the addition function \(+ : V \times V \rightarrow V\) restricts to a function \(+ : U \times U \rightarrow U.\) We can use the same function as the addition function on $U$. 
            \item[(3)] We do the same with scalar multiplication. 
        \end{enumerate}
        Second: We must show (A1) - (M4) hold. We only do (A1) since the rest are similar. To prove (A1), take arbitrary \(u,v \in U\). We need to show that \[u + v = v + u\] in $U$. But since $V$ is a vector space, we know that \[u + v = v + u\] in $V$. This automatically holds.

        \vspace{1em}

        The other parts of the definition of a vecotr space, such as associativity and commutativity, are automatically satisfied for $U$ because they hold on the larger space $V$. Thus, $U$ is a vector space and hence is a subspace of $V$.
    \end{proof}

    \section{Examples of Subspaces}

    \begin{enumerate}
        \item[(a)] Let \(V = \mathbb{R}^4\) and let \[W = \{ (x,y,z,w) \mid x = 3y + 2z \}.\] Then $W$ is a subspace of $V$.
        \begin{proof}
            We need to show that $W$ is a subspace of $V$.
            \begin{enumerate}
                \item[(1)] \((0,0,0,0) \in W\).
                \item[(2)] Assume \(v = (x,y,z,w) \in W\) and \(v' = (x',y',z',w') \in W.\) We need to show that \[x + x' = 3(y + y') + 2(z + z').\] We know that \(v \in W\) implies \[x = 3y + 2z.\] And we know that \(v' \in W\) implies \[x' = 3y' + 2z'.\] Add these two equations together and we get \[x + x' = 3(y + y') + 2(z + z').\]
                \item[(3)] Similar proof for scalar multiplication. 
            \end{enumerate}
        \end{proof} 

        \item[(b)] Recall that \(V = \mathbb{R}^{[0,1]}\) is the vector space of functions from the unit interval \[[0,1] = \{x \in \mathbb{R} \mid 0 \leq x \leq 1\}\] to \(\mathbb{R}.\) Let \(W = \{f : [0,1] \rightarrow \mathbb{R} \mid \text{ $f$ is continuous }\}.\) Then $W$ is a subspace of $V$.
        \begin{proof}
            We need to show that $W$ is a subspace of $V$.
            \begin{enumerate}
                \item[(1)] The zero function \(f(x) = 0\) is continuous. From Calculus, any constant function is continuous.
                \item[(2)] The sum of two continuous functions is continuous. This is from Calculus.
                \item[(3)] If $f$ is continuous, then so is $kf$, for any $k \in \mathbb{R}.$ This is also from Calculus.   
            \end{enumerate}
        \end{proof} 

        \item[(c)] Again, let \(V = \mathbb{R}^{[0,1]}\), and \(U = \{f: [0,1] \rightarrow \mathbb{R} \mid \text{ $f$ is differentiable }\}.\) Then $U$ is a subspace of $V$.
        \begin{proof}
            From Calculus, we know these hold true:
            \begin{enumerate}
                \item[(1)] The zero function \(f(x) = 0\) is differentiable with the derivative of \[f'(x) = 0.\]
                \item[(2)] If $f,g$ are differentiable then so is $f+g$ and \[(f+g)' = f' + g'.\]
                \item[(3)] If $f$ is differentiable and $k$ is a scalar, then $kf$ is differentiable. \[(kf)' = kf'.\]  
            \end{enumerate}
            For example, the derivative of \((0 + 2 \sin(x) - 3 \cos(x))' = 0 + 2 \cos(x) - 3(-\sin(x))\).

            We also know from Calculus that every differentiable function is continuous.
            \[U \subseteq W \subseteq V\] where $U$ is the set of all differentiable functions, $W$ is the set of all continuous functions, and $V$ is the set of all functions.
        \end{proof} 

        \item[(d)] Let \(V = \mathbb{R}^{[0,1]}\) and define \[X = \left\{ f:[0,1] \rightarrow \mathbb{R} \mid \text{ $f$ is differentiable and $f' \left( \frac{1}{2} \right) = 0$}\right\}.\] We claim that $X$ is a subspace of $V$.
        
        \emph{Note.} We already know that $U$, the set of differentiable functions, is a subspace of $V$.
        
        \begin{proof}
            Effectively, it is sufficient to show that $X$ is a subspace of $U$.
            \begin{enumerate}
                \item[(1)] The zero function \(f(x) = 0\) is differentiable since \[f'(x) = 0 \text{ and } f'\left(\frac{1}{2}\right) = 0.\] So \(f \in X\).
                \item[(2)] Given \(f,g \in X\), we must check that \(f + g \in X\). Clearly \(f + g\) is differentiable. We must check that \((f + g)' \left( \frac{1}{2} \right) = 0.\) From Calculus, we have
                \begin{align*}
                    (f+g)'\left(\frac{1}{2}\right) &= f'\left(\frac{1}{2}\right) + g' \left(\frac{1}{2}\right) \\
                    &= 0 + 0 \\
                    &= 0.
                \end{align*}
                So we have \(f + g \in X\).
                \item[(3)] Similar proof as follows for scalar multiplication: \[ (kg)' \left(\frac{1}{2}\right) = k \cdot f' \left(\frac{1}{2}\right) = 0. \] 
            \end{enumerate}
        \end{proof}

        \item[(e)] Let \(V = \mathbb{R}^{\infty}\), the set of infinite sequences of real numbers. \[\mathbb{R}^{\infty} = \{(a_0,a_1,a_2,a_3, \dots) \mid a_0,a_1, \dots \in \mathbb{R}\}.\] Recall that $V$ is a vector space. Let $W \subseteq V$ be the set of \emph{convergent} sequences. From Calculus, we know that some sequences converge and some do not.
        
        Some examples are:
        \begin{itemize}
            \item \(a_i = i \Rightarrow (0,1,2,3,4,5,6,7, \dots)\) \[\lim_{i \to \infty} a_i =\text{ does not exist, so it does not converge}\]
            \item \(b_i = \frac{1}{i} \Rightarrow (\frac{1}{1}, \frac{1}{2}, \frac{1}{3}, \dots)\) \[ \lim_{i \to \infty} b_i = \text{ converges to 0} \]
            \item \(c_i = (-1)^i \Rightarrow (1, -1, 1, -1, 1, -1, \dots)\) does not converge.
            \item \(d_i = 2 + \left( - \frac{1}{2} \right)^i \Rightarrow (3, 1.5, 2.25, 1.875, \dots)\) \[\lim_{i \to \infty} d_i = \text{ converges to 2}\]
        \end{itemize}
        Then $W$ is a subspace of $V$.

        \begin{proof}
            We must show that this is true.
            \begin{enumerate}
                \item[(1)] The zero sequence \(a_i = 0 \Rightarrow (0,0,0,\dots)\) converges to 0.
                \item[(2)] From Calculus, the sum of two convergent sequences converges. In fact, \[\lim_{i \to \infty} (a_i + b_i) = \lim_{i \to \infty} a_i + \lim_{i \to \infty} b_i\]
                \item[(3)] Similar proof to scalar muiltiplication. \[\lim_{i \to \infty} (ka_i) = k \lim_{i \to \infty} a_i\]   
            \end{enumerate}
        \end{proof}
        Let $U$ be the set of sequences that converge to 0. Then $U$ is a subspace of $V$ (and of $W$).

        \item[(f)] A reccurence relation. Consider the Fibonacci sequence \[1,1,2,3,5,8,13,21,34,55,89,144,\dots\] Let \(F_n\) be the \(n\)th element of this sequence. Then, we have the following recurrence:
        \begin{align*}
            \text{base case }  F_0 &= 1 \\
            \text{base case }  F_1 &= 1 \\
            \text{reccurence }  F_{n + 2} &= F_n + F_{n + 1} \qquad \text{for all } n \geq 0.
        \end{align*} 
        If we forget the base cases, we can consider the set of \emph{all} sequences satisfying the reccurence \[U = \{(a_0, a_1, a_2, \dots) \mid \text{ for all } n \geq 0, a_{n + 2} = a_n + a_{n+1}\}\] A sequence of numbers is called a \emph{generalized Fibonacci sequence} if it satisfies this reccurence, i.e. if it is a member of the set $U$.

        \emph{Examples.}
        \begin{align*}
            & 1,2,3,5,8,13, \dots \\
            & 7, -3,4,1,5,6,11,17,28,45, \dots \\
            & 0,0,0,0,0,0,0,0, \dots
        \end{align*} 

        \emph{Claim.} $U$ is a subspace of \(\mathbb{R}^{\infty}\).
        \begin{enumerate}
            \item[(1)] The sequence \(0,0,0, \dots\) is a generealized Fibonacci sequence.
            \item[(2)] $U$ is closed under addition.
            \begin{align*}
                & 1,2,3,5,8,13 \dots \\
                & 7, -3,4,1,5,6, \dots \\
                \cline{1-2}
                & 8,-1,7,6,13,19, \dots
            \end{align*}
            \begin{proof}
                Suppose \(a = (a_0, a_1, a_2, \dots) \in U\) and \(b = (b_0, b_1, b_2, \dots) \in U.\) Then \(a + b = c = (c_0, c_1, c_2, \dots)\) where \(c_i = a_i + b_i.\) 

                We must show \(c \in U\), i.e. we must show that $c$ is a generalized Fibonacci sequence.

                So take an arbitrary \(n \geq 0\). We must show \(C_{n+2} = C_n + C_{n+1}\). Indeed, we have:
                \begin{align*}
                    C_{n+2} &= a_{n+2} + b_{n+2} \\
                            &= (a_n + a_{n+1}) + (b_n + b_{n+1}) \\
                            &= (a_n + b_n) + (a_{n+1} + b_{n+1}) \\
                            &= c_n + c_{n+1}
                \end{align*}
            \end{proof}
            \item[(3)] Closed under scalar multiplication: similar.
        \end{enumerate}
    \end{enumerate}

    \section{Intersection of Subspaces}

    \subsection{Theorem}

    Let $V$ be a vector space over a field $F$. Assume $U$ and $W$ are subspaces of $V$. Then \(U \cap W\) is a subspace of $V$.

    \begin{proof}
        To show that \(U \cap W\) is a subspace, we need to show the three properties.
        \begin{enumerate}
            \item[(1)] We must show that \(0 \in U \cap W\). But by assumption, $U$ is a subspace, so \(0 \in U\). Also, $W$ is a subspace, so \(0 \in W\). By definiton of intersection, we have \(0 \in U \cap W\).
            \item[(2)] We must show that \(U \cap W\) is closed under addition. Consider arbitrary \(v,w \in U \cap W\) and we need to show that \(v + w \in U \cap W\).
            
            Indeed, we have:
            \begin{itemize}
                \item Since \(v \in U \cap W\), we know \(v \in U\).
                \item Since \(w \in U \cap W\), we know \(w \in U\).
                \item Since $U$ is a subspace, it is closed under addition, so \(v + w \in U\).
            \end{itemize}

            Similarly:
            \begin{itemize}
                \item Since \(v \in U \cap W\), we know \(v \in W\).
                \item Since \(w \in U \cap W\), we know \(w \in W\).
                \item Since $W$ is a subspace, it is closed under addition, so \(v + w \in W\).
            \end{itemize}

            From \(v + w \in U\) and \(v + w \in W\), by definition of intersection, we know \[v + w \in U \cap W.\]

            \item[(3)] We must show that \(U \cap W\) is closed under scalar multiplication. So consider arbitrary \(k \in F\) and \(v \in U \cap W\). We must show that \(kv \in U \cap W.\)
            
            Since \(v \in U \cap W\), we have \(v \in U\). Since $U$ is a subspace of $V$, we know that $U$ is closed under scalar multiplication, so $kv \in U$.

            Similarly, since \(v \in U \cap W\), we know \(v \in W.\) Since $W$ is a subspace of $V$, we know that $W$ is closed under scalar multiplication, so $kv \in W$.

            From \(kv \in U\) and \(kv \in W\), it follows that \(kv \in U \cap W\) (by definition of intersection), as desired.
        \end{enumerate}
    \end{proof}

    \subsection{Notations}

    \begin{itemize}
        \item \((x_1, \dots, x_n)\) is called an $n$-tuple.
        \item \((x_i)_{i \in \{1, \dots, n\}}\) is another notation for the same thing. This is called ``family'' notation. But, this notation also works for infinite index sets. \[ (x_i)_{i \in \mathbb{N}} = (x_0, x_1, x_2, \dots) \] \[ \left( \frac{1}{i + 1} \right)_{i \in \mathbb{N}} = \left( 1, \frac{1}{2}, \frac{1}{3}, \frac{1}{4}, \dots \right) \]
        \item More generally, we can use the ``family notation'' for other families of things, not necessarily numbers. 
        \begin{align*}
            U_1, U_2, U_3 && \text{3 subspaces of $V$} \\
            (U_i)_{i \in \{ 1,2,3 \}} && \text{Notation for the same thing.} \\
            (U_i)_{i \in I} && \text{Some family of subspaces.}
        \end{align*}

        \pagebreak
        
        \item Other notations that go along with these:
        \begin{itemize}
            \item Summation. \[\sum_{i \in \mathbb{N}} x_i\]
            \item Product. \[\prod_{i \in \mathbb{N}} x_i \]
            \item Limit. \[\lim_{i \to \infty} x_i\]
            \item Union. \[ \bigcup_{i \in \{1,2,3\}} U_i \]
            \item Intersection. \[ \bigcap_{i \in I} U_i\] 
        \end{itemize}
    \end{itemize}

    \subsection{Theorem}

    Let $V$ be a vector space over a field $F$. Let $I$ be a set. Let \((U_i)_{i \in I}\) be a \emph{family} of subspaces of $V$. Then, \[\bigcap_{i \in I} U_i\] is a subspace of $V$.

    \begin{proof}
        Let \(W = \bigcap_{i \in I} U_i\). To show that $W$ is a subspace of $V$, we must show the three subspace properties.
        \begin{enumerate}
            \item[(1)] We must show that \(0 \in W\). Take an arbitrary \(i \in I\). By assumption, \(U_i\) is a subspace of $V$. Therefore, \(0 \in U_i\).
            
            Since $i$ was arbitrary, we have \(0 \in U_i\) for all $i$, and therefore, by definition of intersection, \(0 \in \bigcap_{i \in I} U_i = W\).

            \item[(2)] Must show $W$ is closed under addition. Take an arbitrary \(v,u \in W\). We must show that \(v + u \in W\) or \(v + u \in \bigcap_{i \in I} U_i\).
            
            Take an arbitrary \(i \in I\). We must show that \(v + u \in U_i\). By assumption, \(v \in W = \bigcap_{i \in I} U_i\), therefore \(v \in U_i\).

            Simiarly, by assumption, \(u \in W = \bigcap_{i \in I} U_i\), therefore \(u \in U_i\).

            Also by assumption, \(U_i\) is a subspace of $V$, therefore it is closed under addition. So \(v + u \in U_i\).

            Since $i$ was arbitrary, we therefore know for all \(i \in I\) that \(v + u \in U_i\). It follows that \(v + u \in \bigcap_{i \in I} U_i\) as desired.

            \item[(3)] Closed under scalar multiplication: similar proof.
        \end{enumerate}
    \end{proof}

    \subsection{Meta-theorem}

    Let $V$ be a vector space over a field $F$. Let $P$ be any property of subspaces of $V$ such that $P$ is closed under arbitrary intersections.

    This means that whenever we have a family of subspaces, \((U_i)_{i \in I}\) of $V$,
    \begin{itemize}
        \item If, for all \(i \in I\), \(U_i\) has the property $P$, then \[\bigcap_{i \in I} U_i\] has the property $P$.
    \end{itemize}
    Then there exists a \emph{smallest} subspace of $V$ with the property $P$.

    \begin{proof}
        Let $P$ be such a property, i.e. a property of subspaces of $V$ that is closed under intersections.

        We want to showt hat there exists a \emph{smallest} subspace $W$ of $V$ with property $P$. Specifically, this means:
        \begin{enumerate}
            \item[(1)] $W$ is a subspace of $V$ and has the property $P$.
            \item[(2)] Whenever $W'$ is a subspace of $V$ that has the property $P$, then $W \subseteq W'$.  
        \end{enumerate}

        To show it, let \((U_i)_{i \in I}\) be the family of \emph{all} subspaces of $V$ satisfying property $P$. Define \[W = \bigcap_{i \in I} U_i. \] We have to show (1) and (2).

        \begin{enumerate}
            \item[(1)] $W$ is a subspace of $V$ by the previous theorem. Also, $W$ has the property $P$ because all $U_i$ have the property $P$ and $P$ is closed under intersections.
            \item[(2)] We must show that $W$ is smallest. So consider any subspace $W'$ with property $P$. We must show that \(W \subseteq W'\).
            
            But the family \((U_i)_{i \in I}\) contains \emph{all} subspaces with property $P$. So \(W' = U_i\) for all some \(i \in I\). Then \[W = \bigcap_{i \in I} U_i \subseteq U_i = W'.\]
        \end{enumerate}
    \end{proof}

    \subsubsection{Example}
    
    Consider \(v_1, v_2, v_3 \in V\).

    There exists a \emph{smallest subspace} $W$ of $V$ such that \(v_1, v_2, v_3 \in W\). We normally call $W$ the \emph{span} of \(v_1, v_2, v_3\).

    \begin{proof}
        By the meta-theorem, the property ``contains \(v_1, v_2\), and \(v_3\)'' is closed under intersections.
    \end{proof}
    
    \section{Sums of Subspaces}

    \subsection{Definition of Sum of Subsets}

    Suppose \(U_1, \dots, U_m\) are subsets of $V$. The \textbf{sum} of \(U_1, \dots, U_m\), denoted \(U_1 + \dots + U_m\), is the set of all possible sums of elements of \(U_1, \dots, U_m\). More precisely, \[U_1 + \dots + U_m = \{u_1 + \dots + u_m \mid u_1 \in U_1, \dots, u_m \in U_m\}.\]

    \subsubsection{Example}

    For \(U = \{(x,0,0) \in \textbf{F}^3 \mid x \in \textbf{F}\}\) and \(W = \{(0,y,0) \in \textbf{F}^3 \mid y \in \textbf{F}\}\), we have \[U + W = \{(x,y,0) \mid x,y \in \textbf{F}\}.\]

    For \(U = \{(x,x,y,y) \in \textbf{F}^4 \mid x,y \in \textbf{F}\}\) and \(W = \{(x,x,x,y) \in \textbf{F}^4 \mid x,y \in \textbf{F}\}\), then \[U + W = \{(x,x,y,z) \in \textbf{F}^4 \mid x,y,z \in \textbf{F}\}.\]

    \subsection{Sum of subspaces is the smallest containing subspace}

    Suppose \(U_1, \dots, U_m\) are subspaces of $V$. Then \(U_1 + \dots + U_m\) is the smallest subspace of $V$ containing \(U_1, \dots, U_m\).

    \begin{proof}
        We know that \(0 \in U_1 + \dots + U_m\) and \( U_1 + \dots + U_m \) is closed under addition and scalar multiplication. Thus, \(U_1 + \dots + U_m\) is a subspace of $V$.

        \(U_1, \dots, U_m\) are contained in \(U_1 + \dots + U_m\), and every subspace of $V$ containing \(U_1, \dots, U_m\) contains \(U_1 + \dots + U_m\).

        Thus, \(U_1 + \dots + U_m\) is the smallest subspace of $V$ containing \(U_1, \dots, U_m\).
    \end{proof}

    \emph{Note.} Sums of subspaces in theory of vector spaces are analogous to unions of subsets in set theory. Given two subspaces of a vector space, the smallest subspace containing them is their sum. Analogously, given two subsets of a set, the smallest subset containing them is their union.

    \section{Direct Sums}

    Suppose \(U_1, \dots, U_m\) are subspaces of $V$. Every element of \(U_1 + \dots + U_m\) can be written in the form \[u_1 + \dots + u_m\] where each \(u_j\) is in $U_j$.

    \subsection{Definition of direct sum}

    \begin{itemize}
        \item The sum \(U_1 + \dots + U_m\) is called a \emph{direct sum} if each element of \(U_1 + \dots + U_m\) can eb written in only one way as a sum \(u_1 + \dots + u_m\), where each \(u_j\) is in \(U_j\).
        \item If \(U_1 + \dots + U_m\) is a direct sum, then \(U_1 \oplus \dots \oplus U_m\) denotes \(U_1 + \dots + U_m\), with the \(\oplus\) notation serving as an indication that this is a direct sum. 
    \end{itemize}

    \subsubsection{Examples}

    For \(U = \{(x,y,0) \in \textbf{F}^3 \mid x,y \in \textbf{F}\}\) and \(W = \{(0,0,z) \in \textbf{F}^3 \mid z \in \textbf{F}\}\), then \[\textbf{F}^3 = U \oplus W.\]

    Suppose \(U_j = \{(0,0,0,\dots, j) \in \textbf{F}^n \mid x \in \textbf{F}\}\). Then \[\textbf{F}^n = U_1 \oplus \dots \oplus U_n.\]

    \textbf{Non-example:} Let \(U_1 = \{(x,y,0) \in \textbf{F}^3 \mid x,y \in \textbf{F}\}\), \(U_2 = \{(0,0,z) \in \textbf{F}^3 \mid z \in \textbf{F}\}\), \(U_3 = \{(0,y,y) \in \textbf{F}^3 \mid y \in \textbf{F}\}\). Then, \(U_1 + U_2 + U_3\) is not a direct sum. 

    \begin{proof}
        Clearly \(\textbf{F}^3 = U_1 + U_2 + U_3\) because every vector \((x,y,z)\in \textbf{F}^3\) can be written as \[(x,y,z) = (x,y,0) + (0,0,z) + (0,0,0).\] This is not equal to the direct sum because the vector \((0,0,0)\) can be written in two different ways. \[(0,0,0) = (0,0,0) + (0,0,0) + (0,0,0)\] and \[(0,0,0) = (0,1,0) + (0,0,1) + (0,-1,-1).\]
    \end{proof}

    \subsection{Condition for a direct sum}
    
    Suppose \(U_1, \dots, U_m\) are subspaces of $V$. Then \(U_1 + \dots + U_m\) is a direct sum if and only if the only way to write 0 as a sum \(u_1 + \dots + u_m\), where each \(u_j\) is in \(U_j\), is by taking each \(u_j\) equal to 0.

    \begin{proof}
        Suppose \(U_1 + \dots + U_m\) is a direct sum. Then by definition, the only way to write 0 as a sum is by taking each \(u_j\) equal to 0. To show that \(U_1 + \dots + U_m\) is a direct sum, let \(v \in U_1 + \dots + U_m.\) We can write \[v = u_1 + \dots + u_m\] for some \(u_1 \in U_1, \dots, u_m \in U_m.\) To show that this is unique, suppose we also have \[v = v_1 + \dots + v_m\] where \(v_1 \in U_1, \dots, v_m \in U_m\). Subtracting them, we have \[0 = (u_1 - v_1) + \dots + (u_m - v_m).\] This implies that \(u_j = v_j\). 
    \end{proof}

    \subsection{Direct sum of two subspaces}

    Suppose $U$ and $W$ are subspaces of $V$. Then $U+W$ is a direct sum if and only if \(U \cap W = \{0\}\).

    \begin{proof}
        Suppose that $U+W$ is a direct sum. If \(v \in U \cap W\), then \(0 = v + (-v)\) where \(v \in U\) and \(-v \in W\). This implies \(v = 0\) since it is unique. 

        To prove the other direction, suppose \(U \cap W = \{0\}\). To prove that \(U + W\) is a direct sum, suppose \(u \in U, w \in W\), and \[0 = u + w.\] By the previous theorem, we know that \(u = w = 0.\) The equation above implies \(u = -w \in W,\) so \(u \in U \cap W\) and \(u = w = 0\) is true.
    \end{proof}

    \emph{Note.} Sums of subspaces are analogous to unions of subsets. Similarly, direct sums of subspaces are analogous to disjoint unions of subsets. No two subspaces of a vector space can be disjoint, because both contain 0. So disjointness is replaced, at least in the case of two subspaces, with the requirement that the intersection equals \(\{0\}\).


    \end{document}