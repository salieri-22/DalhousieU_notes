\documentclass[11pt]{article}

\usepackage{mathtools}
\usepackage{amssymb}
\usepackage{amsmath}
\usepackage{amsthm}
\usepackage{hyperref}
\usepackage{microtype}
\usepackage{graphicx}
\graphicspath{ {./img/} }

\setlength{\parindent}{0cm}
\let\emptyset\varnothing

\title{\textbf{CSCI/MATH 2113 Discrete Structures} \\ 7.1 Relations Revisited: Properties of Relations}
\author{Alyssa Motas}

\begin{document}

    \maketitle

    \pagebreak

    \tableofcontents

    \pagebreak

    \section{Recall: Binary Relation}

    For sets $A, B$, any subset of \(A \times B\) is called a (\emph{binary}) \emph{relation} from $A$ to $B$. Any subset of \(A \times A\) is called a (\emph{binary}) \emph{relation} on $A$.

    \section{Properties of Relations}

    \subsection{Reflexive property}

    A relation $R$ on a set $A$ is called \emph{reflexive} if for all \(x \in A\), \((x,x) \in R\). This means that each element $x$ of $A$ is related to itself. 

    \vspace{1em}

    \emph{Example.} For \(A = \{1,2,3\}\) and \(R = \{(1,1), (1,2), (1,3), (2,2), (2,3)\}\), we know it is not reflexive since \((3,3) \notin R\).

    \vspace{1em}

    \emph{Remark.} A relation $R$ on $A$ is reflexive if and only if \(\{(a,a) \mid a \in A\} \subseteq R\).

    \vspace{1em}

    \emph{Counting.} Let $A$ be a set with $n$ elements. How many relations on $A$ are reflexive? There are \(2^{n^2}\) relations on $A$ (2 choices for each of the $n^2$ pairs in $A \times A$). In a reflexive relation, $n$ of these pairs are decided. Hence, we have \(n^2 - n\) chocies to make and so there are  \[2^{n^2 - 2}\] reflexive relations.

    \subsection{Symmetric property}

    A relation $R$ on a set $A$ is called \emph{symmetric} if \((x,y) \in R \Rightarrow (y,x) \in R\), for all, \(x,y \in A\).

    \vspace{1em}

    \emph{Counting.} If \(|A| = n \leq 0\), how many relations on $A$ are symmetric? Write \(A \times A = A_1 \cup A_2\) where
    \begin{align*}
        A_1 &= \{(a_i, a_i) \mid 1 \leq i \leq n\} \\
        A_2 &= \{(a_i, a_j) \mid 1 \leq i, j \leq n, i \neq j\}.
    \end{align*}
    We have \[|A_1| = n \qquad |A_2| = n^2 - n.\] For each element of \(A_1\) and for half of the elements of \(A_2\), we choose whether it belongs to $R$. Thus, intotal, there are \[2^n \cdot 2^{\frac{n^2-n}{2}} = 2^{\frac{n^2+n}{2}}\] such relations. In counthing those relations on $A$ that are both reflexive and symmetric, we have only one choice for each ordered pair in \(A_1\). So we have \[2^{\frac{n^2 - n}{2}}\] relations on $A$ that are both reflexive and symmetric.

    \subsection{Transitive property}

    For a set $A$, a relation $R$ on $A$ is called \emph{transitive} if, for all \(x,y,z \in A, (x,y), (y,z) \in R \Rightarrow (x,z) \in R\). So if $x$ ``is related to'' $y$, and $y$ ``is related to'' $z$, we want $x$ ``related to'' $z$, with $y$ playing the role of ``intermediary.''

    \vspace{1em}

    \emph{Counting.} There is no known general formula for the total number of transitive relations on a finite set.

    \subsection{Antisymmetric property}

    Given a relation $R$ on a set $A$, $R$ is called \emph{antisymmetric} if for all \(a,b \in A\), (\(a R b\) and \(b R a\)) \(\Rightarrow a = b.\) Here, the only way we can have both $a$ ``related to'' $b$ and $b$ ``related to'' $a$ is if $a$ and $b$ are one and the same element from $A$.

    \vspace{1em}

    \emph{Examples.} Let \(A = \mathbb{Z}\) and \(R = \leq\) is an antisymmetric relation.

    \vspace{1em}

    Antisymmetric is different than ``not symmetric.'' Let \(A = \{1,2,3\}\) and \(R = \{(1,2), (2,1), (2,3)\}\). Then, $R$ is \emph{not} symmetric and it is \emph{not} antisymmetric.

    \vspace{1em}

    \emph{Counting.} How relations that are antisymmetric? Suppose that \(|A| = n > 0\). By the rule of product, the number of antisymmetric relations are \[(2^n)(3^{\frac{n^2-n}{2}}).\]

    \section{Partial order}

    A relation $R$ on a set $A$ is called a \emph{partial order}, or a \emph{partial ordering relation}, if $R$ is reflexive, antisymmetric, and transitive.

    \subsection{Example}

    \begin{itemize}
        \item \(\leq\) on \(\mathbb{N}\). Partial order and total implies total order.
        \item \(\subseteq\) on \(\mathcal{P}(S)\) is not total order.
    \end{itemize}

    \section{Total order}

    A relation $R$ on a set $A$ is a \emph{total order} if $R$ is a partial order and for every \(x,y \in A,\) we have \((x,y) \in R\) or \((y,x) \in R\).

    \section{Equivalence relation}

    An \emph{equivalence relation} on $A$ is a relation that is reflexive, symmetric, and transitive.

\end{document}