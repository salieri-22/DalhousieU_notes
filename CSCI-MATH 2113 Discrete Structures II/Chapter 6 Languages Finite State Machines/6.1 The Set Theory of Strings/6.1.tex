\documentclass[11pt]{article}

\usepackage{mathtools}
\usepackage{amssymb}
\usepackage{amsmath}
\usepackage{amsthm}
\usepackage{hyperref}
\usepackage{microtype}
\usepackage{graphicx}
\graphicspath{ {./img/} }

\setlength{\parindent}{0cm}
\let\emptyset\varnothing

\title{\textbf{CSCI/MATH 2113 Discrete Structures} \\ 6.1 Language: The Set Theory of Strings}
\author{Alyssa Motas}

\begin{document}

    \maketitle

    \pagebreak

    \tableofcontents

    \pagebreak

    \section{Definition of an alphabet}

    An \emph{alphabet} is a finite nonempty set. We write \(\Sigma\) for an alphabet and we sometimes call the elements of \(\Sigma\) \emph{letters}. For example, we may have \(\Sigma = \{0,1\}\) or \(\Sigma = \{a,b,c,d,e\}\).

    \section{Powers of an alphabet}

    If \(\Sigma\) is an alphabet and \(n \in \mathbb{Z}^+\), we define the \emph{powers} of \(\Sigma\) recursively as follows:
    \begin{enumerate}
        \item \(\Sigma^1 = \Sigma;\) and 
        \item \(\Sigma^{n+1} = \{xy \mid x \in \Sigma, y \in \Sigma^n\}\), where $xy$ denotes the juxtaposition of $x$ and $y$.
    \end{enumerate} 

    \subsection{Example}

    Let \(\Sigma\) be an alphabet. With \(\Sigma = \{0,1\}\), we find that \[\Sigma^2 = \{00, 01, 10, 11\} \text{ and } |\Sigma^2| = |\Sigma|^2 = 2^2 \text{ two-symbol strings}.\] In general, we have \(|\Sigma^n| = |\Sigma|^n\).

    \section{Empty string}

    For an alphabet \(\Sigma\), we define \(\Sigma^0 = \{\lambda\}\), where \(\lambda\) denotes the \emph{empty string}. That is, the string consisting of \emph{no} symbols taken from \(\Sigma\). Note that even though \(\lambda \notin \Sigma\), we do have \(\emptyset \subseteq \Sigma\). Also, \(\{\lambda\} \neq \emptyset\) because \(|\{\lambda\}| = 1 \neq 0 = |\emptyset|\).

    \section{Union of alphabets}

    If \(\Sigma\) is an alphabet, then 
    \begin{enumerate}
        \item[(a)] \(\Sigma^+ = \bigcup_{n=1}^{\infty} \Sigma^n = \bigcup_{n \in \mathbb{Z}^+} \Sigma^n;\)
        \item[(b)] \(\Sigma^* = \bigcup_{n = 0}^{\infty} \Sigma^n.\)
    \end{enumerate}
    The difference between (a) and (b) is that \(\lambda \in \Sigma^n\) only when \(n = 0\). Also, \(\Sigma^* = \Sigma^+ \cup \Sigma^0.\)

    We shall also refer to the elements of \(\Sigma^+\) or \(\Sigma^*\) as \emph{words} and sometimes as \emph{sentences}. Finally, we note that even though the sets \(\Sigma^+\) and \(\Sigma^*\) are \emph{infinite}, the elements of these sets are \emph{finite} strings of symbols.

    \subsection{Example}

    \begin{itemize}
        \item For \(\Sigma = \{0,1\}\) the set \(\Sigma^*\) consists of all finite strings (binary words) of 0's and 1's together with the empty string.
        \item If \(\Sigma = \{+, \times, 0,1, \dots, 9, (,), \hspace{1em}\}\) we have \[((14 + 12) \times 3) \times 1009 \in \Sigma^* \text{ or } )+(x)1+(\times 3) \in \Sigma^*.\]
    \end{itemize}

    \section{Equality of sets}

    If \(w_1,w_2 \in \Sigma^+\), then we may write \[ w_1 = x_1 x_2 \dots x_m \text{ and } w_2 = y_1 y_2 \dots y_n \] for \(m,n \in \mathbb{Z}^+\) and \(x_1, x_2, \dots, x_m, y_1, y_2, \dots, y_n \in \Sigma\). We say that the strings \(w_1\) and \(w_2\) are \emph{equal}, and we write \(w_1 = w_2\), if \(m = n\), and \(x_i = y_i\) for all \(1 \leq i \leq m\).

    \section{Length}

    Let \(w = x_1 x_2 \dots x_n \in \Sigma^+\), where \(x_i \in \Sigma\) for each \(1 \leq i \leq n\). We define the \emph{length} of $w$, which is denoted by \(||w||\), as the value $n$. For the case of \(\lambda\), we have \(||\lambda|| = 0\).

    \section{Concatenation}

    Let \(x,y \in \Sigma^+\) with \(x = x_1 x_2 \dots x_m\) and \(y = y_1 y_2 \dots y_n\), so that each \(x_i\), for \( 1 \leq i \leq m \), and each \(y_j\), for \(1 \leq j \leq n\), is in \(\Sigma\). The \emph{concatenation} of $x$ and $y$, which we write as $xy$, is the string \[ x_1 x_2 \dots x_m y_1 y_2 \dots y_n. \] The concatenation of $x$ and $\lambda$ is \(x \lambda = x_1 x_2 \dots x_m \lambda = x_1 x_2 \dots x_m = x\), and the concatenation of $\lambda$ and $x$ is \(\lambda x = \lambda x_1 x_2 \dots x_m = x_1 x_2 \dots x_m = x.\) Finally, the concatenation of \( \lambda \) and \(\lambda\) is \(\lambda \lambda = \lambda.\)

    \vspace{1em}

    Here, we have defined a closed binary operation on \(\Sigma^*\) (and \(\Sigma^+\)). This operation is associative but not commutative unless \(|\Sigma| = 1\). The \(\lambda\) is also the identity for the operation of concatenation. We also have \[ ||xy|| = ||x|| + ||y||, \qquad \text{ for all }x,y \in \Sigma^*. \]

    \section{Powers of a string}

    For each \(x \in \Sigma^*\), we define the \emph{powers} of $x$ by \( x^0 = \lambda, x^1 = x, x^2 = xx, x^3 = xx^2, \dots, x^{n+1} = xx^n, \dots, \) where \( n \in \mathbb{N} \).

    \section{Proper prefix and suffix}

    If \(x,y \in \Sigma^*\) and \(w = xy\), then the string $x$ is called a \emph{prefix} of $w$, and if \(y \neq \lambda\), then $x$ is said to be a \emph{proper prefix}. Similarly, the string $y$ is called a \emph{suffix} of $w$; it is a \emph{proper suffix} when \(x \neq \lambda\).

    \vspace{1em}

    In general, for an alphabet \(\Sigma\), if \(n \in \mathbb{Z}^+\) and \(x_i \in \Sigma\), for all \( 1 \leq i \leq n \), then each of \(\lambda, x_1, x_1x_2, x_1 x_2 x_3, \dots,\) and \(x_1 x_2 x_3 \dots x_n\) is a prefix of the string \( x = x_1 x_2 x_3 \dots x_n \). And \(\lambda, x_n, x_{n-1} x_n, x_{n-2} x_{n-1} x_n, \dots,\) and \( x_1 x_2 x_3 \dots x_n \) are all suffixes of $x$. So, $x$ has \(n+1\) prefixes, $n$ of which are proper, and the situation is the same for suffixes.

    \section{Substring}

    If \(x,y,z \in \Sigma^*\) and \(w = xyz\), then $y$ is called a \emph{substring} of $w$. When at least one of $x$ and $z$ is different from \(\lambda\) (so that $y$ is different from $w$), we call $y$ a \emph{proper substring or subword.}

    \section{Language}

    For a given alphabet \(\Sigma\), any subset of \( \Sigma^* \) is called a \emph{language} over \(\Sigma\). This includes the subset \(\emptyset\), which we call the \emph{empty language}.

    \subsection{Example}

    With \( \Sigma = \{0,1\} \), the sets \[A = \{0,01,001\}\] and \[ B = \{0,01,001,0001, \dots\} \] are examples of languages over \(\Sigma\).

\end{document}