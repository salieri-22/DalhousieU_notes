\documentclass[11pt]{article}

\usepackage{mathtools}
\usepackage{amssymb}
\usepackage{amsmath}
\usepackage{amsthm}
\usepackage{hyperref}
\usepackage{microtype}

\setlength{\parindent}{0cm}
\let\emptyset\varnothing

\title{\textbf{CSCI/MATH 2113 Discrete Structures} \\ 5.3 Onto Functions: Stirling Numbers of the Second Kind}
\author{Alyssa Motas}

\begin{document}

    \maketitle

    \pagebreak

    \tableofcontents

    \pagebreak

    \section{Surjective Functions}
    \subsection{Definition}

    \(f: A \rightarrow B\) is \emph{onto} or \emph{surjective} if \(\forall b \in B, \exists a \in A\) such that \(f(a)=b.\)

    \vspace{1em}

    \emph{Remark.} A function is onto if its range is equal to the codomain. 

    \vspace{1em}

    If \(A,B\) are finite sets, then for an onto function \(f: A \rightarrow B\) to possibly exist we must have \(|A| \leq |B|\). 

    \subsection{Examples}
    \begin{itemize}
        \item For the function \(f: \mathbb{R} \rightarrow \mathbb{R}\) defined by \(f(x)=x^3\) we have: if \(x \in \mathbb{R}\), then \(f(\sqrt[3]x) = (\sqrt[3]x)^3 = x.\) This means that $f$ is onto.
        \item For the function \(g: \mathbb{R} \rightarrow \mathbb{R}\) defined by \(g(x)=x^2\) we have: no such \(y \in \mathbb{R}\) that satisfies \(y^2 = -1.\) For instance, \[g(x) = x^2 = -9 \Rightarrow r = 3i, -3i \in \mathbb{C}.\] Therefore, $g$ is not onto.
    \end{itemize}

    \subsection{Counting}

    Suppose that \(A = \{x,y,z\}\) and \(B = \{1,2\}\). How many \(f: A \rightarrow B\) are onto? 
    \begin{proof}
        The function \(f: A \rightarrow B\) is not onto if and only if \(f(a) = 1\) for all \(a \in A\) or \(f(a)=2\) for all \(a \in A\). Hence, the number we seek is \[|B|^{|A|} - 2 = 2^3 - 2 = 6.\]
    \end{proof}
    In general, if $A$ and $B$ are sets with \(|A| = m\) and \(|B| = n\), then this quantity is \[ \sum_{k = 0}^{n} (-1)^k \binom{n}{n - k} (n-k)^m \] onto functions from $A$ to $B$.

    \section{Stirling Numbers}

    \subsection{A combinatorial interpretation}

    In how many ways can you distribute 4 objects into 3 labelled cotainers with no container empty?
    We just need to count the number of surjections from $A$ to $B$. By the previuos formula, this is \[\sum_{k = 0}^{3} (-1)^k \binom{3}{3-k} (3-k)^4 = 36.\] What if we had the same situation but with containers that aren't labelled? The number of such distributions is \[S(m,n) = \frac{1}{n!} \sum_{k=0}^n (-1)^k \binom{n}{n-k} (n-k)^m\] where the factor of \(\frac{1}{n!}\) corrects for the fact that certain distributions are equivalent.

    So \(S(m,n)\) is the number of ways to distribute $m$ objects into $n$ identical containers with no container left empty. 

    \subsection{Definition}

    \(S(m,n)\) is a \emph{Stirling number} of the 2nd kind.
    
\end{document}