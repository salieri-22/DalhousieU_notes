\documentclass[11pt]{article}

\usepackage{mathtools}
\usepackage{amssymb}
\usepackage{amsmath}
\usepackage{amsthm}
\usepackage{hyperref}
\usepackage{microtype}

\setlength{\parindent}{0cm}
\let\emptyset\varnothing

\title{\textbf{CSCI/MATH 2113 Discrete Structures} \\ 5.4 Special Functions}
\author{Alyssa Motas}

\begin{document}

    \maketitle

    \pagebreak

    \tableofcontents

    \pagebreak

    \section{Binary operations}
    \subsection{Definition}

    For any nonempty sets \(A,B\), any function \(f: A \times A \rightarrow B\) is called \emph{binary operation} on $A$. If \(B \subseteq A\), then the binrary operation is said to be \emph{closed} (on $A$). (When \(B \subseteq A\) we may also say that $A$ is \emph{closed under f}.)

    \vspace{1em}

    \emph{Remark.} Similarly, \(f : A^n \rightarrow B\) is an \emph{n-ary} operation on $A$. When $n=1$, the operation is \emph{unary} or \emph{monary}.

    \subsection{Examples of Binary Operations}
    \begin{itemize}
        \item For \(f: \mathbb{Z} \times \mathbb{Z} \rightarrow \mathbb{Z}\) defined by \(f(a,b) = a - b\), it is a closed binary operation on \(\mathbb{Z}\).
        \item For \(g: \mathbb{Z} \times \mathbb{Z} \rightarrow \mathbb{R}\) defined by \(g(a,b) = a^b\), it is a non-closed binary operation.
        \item For \(h: \mathbb{Z} \times \mathbb{Z} \rightarrow \mathbb{R}\) defined by \(h(a,b) = a + b\), it is a binary operation.
        \item For \(j: \mathcal{P}(A) \times \mathcal{P}(A) \rightarrow \mathcal{P}(A)\) defined by \(j(S,T) = S \cup T\), it is a closed binary operation.
        \item For \(k: \mathcal{P}(A) \rightarrow \mathcal{P}(A)\) defined by \(k(S) = S^c\), it is a closed unary operation.
    \end{itemize}

    \subsection{Commutativity and Associativity}
    Let \(f: A \times A \rightarrow B\); that is, $f$ is a binary operation on $A$.
    \begin{enumerate}
        \item[(a)] $f$ is said to be \emph{commutative} if \(f(a,b) = f(b,a)\) for all \((a,b) \in A \times A.\)
        \item[(b)] When \(B \subseteq A\) (that is, when $f$ is closed), $f$ is said tot be $associative$ if for all \(a,b,c \in A, f(f(a,b),c) = f(a,f(b,c)).\)  
    \end{enumerate}

    \subsection{Examples of Commutativity and Associativity}
    
    Binary operations that are both commutative and associative:
    \begin{itemize}
        \item + on \(\mathbb{Z}\): \(n + m = m + n\) and \(n + (m + r) = (n + m) + r\)
        \item \(\times\) on \(\mathbb{Z}\)
        \item \(\cup\) on \(\mathcal{P}(A)\)
    \end{itemize}

    Binary operations that are associative but not commutative:
    \begin{itemize}
        \item \(\times\) on \(Mat_{n \times n}(\mathbb{R})\) which is the multiplication of $n \times n$ real matrices.
    \end{itemize}


    Binary operations that are both not commutative and associative:
    \begin{itemize}
        \item \(-\) on \(\mathbb{Z}\): \[2 - 3 = -1 \neq 1 = 3 - 2\] \[((3-3) - 2) = -2 \neq 2 = 3 - (3 - 2) \]
    \end{itemize}

    \subsection{Symmetry}

    Suppose that \(f: A \times A \rightarrow A\) is a binary operation where \(A = \{a_1, \dots, a_n\}\). We can represent $f$ using a \emph{table}.
    \begin{center}
        \begin{tabular}{| c | c | c | c | c |} \hline
            $f$      & $a_1$        & $a_2$        & $\dots$ & $a_n$        \\ \hline
            $a_1$    & $f(a_1,a_1)$ & $f(a_1,a_2)$ &         &              \\ \hline
            $a_2$    & $f(a_2,a_1)$ &              &         &              \\ \hline
            $\vdots$ &              &              &         &              \\ \hline
            $a_n$    &              & $f(a_n,a_2)$ &         & $f(a_n,a_n)$ \\ \hline
            
        \end{tabular}
    \end{center}
    If the operation is commutative, then the table is \emph{symmetric.}

    Now let \(f: \{a,b,c\} \times \{a,b,c\} \rightarrow \{a,b,c\}\) be defined by the table:
    \begin{center}
        \begin{tabular}{| c | c | c | c |} \hline
            $f$ & $a$ & $b$ & $c$ \\ \hline
            $a$ & $b$ & $a$ & $a$ \\ \hline
            $b$ & $a$ & $c$ & $a$ \\ \hline
            $c$ & $a$ & $a$ & $c$ \\ \hline
        \end{tabular}
    \end{center}
    Here we have \[f(a,f(b,c)) = f(a,a) = b \neq a = f(a,c) = f(f(a,b),c)\] so the operation is \emph{not associative} but it is commutative since the table is symmetric.

    \pagebreak

    \section{Identity Element}

    \subsection{Definition}

    Let \(f: A \times A \rightarrow B\) be a binary operation on $A$. An element \(x \in A\) is called an \emph{identity} (or \emph{identity element}) for $f$ if \(f(a,x) = f(x,a) = a\), for all \(a \in A\).

    \subsection{Examples}

    \begin{itemize}
        \item 0 for + on \(\mathbb{Z}\) since \[a + 0 = 0 + a = a\] for all \(a \in \mathbb{Z}\).
        \item \(I_n\) (identity matrix) of $x$ on \(Mat_{n \times n} (\mathbb{R})\).
        \item \(\emptyset\) for \(\cup\) on \(\mathcal{P}(A)\).
        \item $A$ for \(\cap\) on \(\mathcal{P}(A)\).
    \end{itemize}

    \subsection{Theorem}

    Let \(f : A \times A \rightarrow B\) be a binary operation. If $f$ has an identity, then that identity is unique.
    \begin{proof}
        If $f$ has more than one identity, let \(x_1, x_2 \in A\) with
        \begin{align*}
            f(a, x_1) = a = f(x_1,a), && \text{for all } a \in A, \\
            f(a, x_2) = a = f(x_2,a), && \text{for all } a \in A.
        \end{align*}
        Consider \(x_1\) as an element of $A$ and $x_2$ as an identity. Then \(f(x_1,x_2) = x_1\). Now reverse the roles of \(x_1\) and \(x_2\), that is, consider \(x_2\) as an element of $A$ and $x_1$ as an identity. We find that \(f(x_1,x_2) = x_2\). Consequently, \(x_1 = x_2\), and $f$ has at most one identity.
    \end{proof}
    


\end{document}