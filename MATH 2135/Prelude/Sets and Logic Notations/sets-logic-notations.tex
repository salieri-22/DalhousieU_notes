\documentclass[11pt]{article}

\usepackage{amsmath}
\usepackage{amssymb}
\usepackage{hyperref}

\setlength{\parindent}{0cm}
\let\emptyset\varnothing

\title{\textbf{MATH 2135 Linear Algebra} \\ Sets and Logic Notations}
\author{Alyssa Motas}

\begin{document}

    \maketitle

    \pagebreak

    \tableofcontents

    \pagebreak

    \section{Sets}

    \subsection{Finite sets}

    A set is an unordered collection of things. A finite set would look something like \(\{1,2,3\}\). ``Unordered" means that the order does not matter, i.e. \(\{1,2,3\}\) and \(\{2,3,1\}\) are the same set. 

    \subsection{Infinite sets}

    An example of an infinite set is the set of natural numbers. \[\mathbb{N} = \{0,1,2,3,\dots\}\] and its respective \emph{set comprehension notation} would look like \[\mathbb{N} = \{ x \mid \text{ x is a natural number }\}.\] Another example of an inifnite set would be \[\{x \in \mathbb{N} \mid \text{ x is prime }\} = \{2,3,5,7,11,13,17, \dots\}.\]

    \subsection{Membership}

    The notation \(\in\) implies membership such as ``$x$ is an element of $A$'' and conversely, the notation \(\notin\) implies ``$x$ is not an element of $A$.''

    \subsection{Equality}

    Two sets $A$ and $B$ are \emph{equal} if they have the same elements. \[A = B \Leftrightarrow (\forall x, x \in A \Leftrightarrow x \in B).\] 

    We say that $A$ is a \emph{subset} of $B$, in symbols \(A \subseteq B\), if all elements of $A$ are elements of $B$. \[A \subseteq B \Leftrightarrow (\forall x, x\in A \Rightarrow x \in B).\]

    \subsection{Empty set}

    The empty set (\(\emptyset\)) is the set with no elements.

    \subsection{Cartesian product}

    If $A$ and $B$ are sets, we define the \emph{cartesian product} of $A$ and $B$, in symbols \(A \times B\), as \[A \times B = \{(x,y) \mid x \in A \text{ and } y \in B\}.\] Note that \((x,y)\) is a \emph{pair} or \emph{2-tuple} which is an \emph{ordered pair}, i.e. \((1,2) \neq (2,1)\). 

    \pagebreak

    \section{Logic Notations}

    \subsection{Propositional logic (Boolean logic)}

    A \emph{proposition} is a statement that can be true or false.

    \vspace{1em}

    Let $P$ and $Q$ be propositions.

    \vspace{1em}

    \begin{tabular}{| c | c | c |} \hline
        $P$ & $Q$ & $P$ and $Q$ \\ \hline
        T   &  T  &      T      \\ \hline 
        T   &  F  &      F      \\ \hline 
        F   &  T  &      F      \\ \hline 
        F   &  F  &      F      \\ \hline 
    \end{tabular} \hspace{0.5em}
    \begin{tabular}{| c | c | c |} \hline
        $P$ & $Q$ & $P$ or $Q$  \\ \hline
        T   &  T  &      T      \\ \hline 
        T   &  F  &      T      \\ \hline 
        F   &  T  &      T      \\ \hline 
        F   &  F  &      F      \\ \hline 
    \end{tabular} \hspace{0.5em}
    \begin{tabular}{| c | c | c |} \hline
        $P$ & $Q$ & $P \Rightarrow Q$ \\ \hline
        T   &  T  &      T            \\ \hline 
        T   &  F  &      F            \\ \hline 
        F   &  T  &      T            \\ \hline 
        F   &  F  &      T            \\ \hline 
    \end{tabular} \hspace{0.5em}
    \begin{tabular}{| c | c | c |} \hline
        $P$ & $Q$ & $P \Leftrightarrow Q$ \\ \hline
        T   &  T  &      T                \\ \hline 
        T   &  F  &      F                \\ \hline 
        F   &  T  &      F                \\ \hline 
        F   &  F  &      T                \\ \hline 
    \end{tabular}

    \vspace{0.5em}

    \begin{center}
        \begin{tabular}{| c | c |} \hline
            $P$ & $\neg P$  \\ \hline
            T   &  F       \\ \hline 
            F   &  T       \\ \hline 
        \end{tabular}
    \end{center}

    \subsection{Predicate logic}

    A \emph{predicate} is a proposition that depends on some ``thing'' $x$.

    \begin{align*}
        P(x) &= \text{``$x$ is a prime number''} \\
        P(5) &= \text{``5 is a prime number'' \emph{TRUE}} \\
        P(6) &= \text{``6 is a prime number'' \emph{FALSE}} \\
        \vdots \\
        Q(x,y)  &\rightarrow \text{``x is greater or equal to y''} \\
        Q(3,7)  &\rightarrow \text{\emph{FALSE}} \\
        Q(7,7)  &\rightarrow \text{\emph{TRUE}} \\
        Q(19,7) &\rightarrow \text{\emph{TRUE}} \\
        \vdots
    \end{align*}

    \subsection{Quantifiers}

    \subsubsection{Universal Quantifier}

    If $P(x)$ is a predicate, then ``for all $x$, $P(x)$'' is a proposition that is either true or false. 

    \pagebreak

    \emph{Example.} Let \(A = \{3,5,7,8,11\}\), \(P(x) = \)``$x$ is prime,'' \(Q(x) = \)``x is even.''
    \begin{itemize}
        \item For all \(x \in A, P(x)\) \(\rightarrow\) \emph{FALSE.} This is because \(P(3),P(5),P(7),P(11)\) are all true but \(P(8)\) is false.
        \item For all \(x \in A\), \((x \leq 7 \Rightarrow P(x))\) \(\rightarrow\) \emph{TRUE.}
        \begin{center}
            \begin{tabular}{|c | c | c | c |} \hline
                $x$ & $x \leq 7$ & $P(x)$ & $x \leq 7 \Rightarrow P(x)$ \\ \hline
                3   & T          & T      & T                           \\ \hline
                5   & T          & T      & T                           \\ \hline
                7   & T          & T      & T                           \\ \hline
                8   & F          & F      & T                           \\ \hline
                11  & F          & T      & T                           \\ \hline
            \end{tabular}
        \end{center}
    \end{itemize}

    The notation for ``for all'' is \(\forall\).

    \subsubsection{Existential Quantifer}

    The notation for ``there exists'' is \(\exists\).

    \vspace{1em}

    \emph{Example.} Let \(A = \{3,5,7,8,11\}\).
    \begin{itemize}
        \item There exists an \(x \in A\) such that \(P(x)\). \emph{TRUE}
        \item There exists \(x \in A\) such that \(x \leq 7\) and \(P(x)\). \emph{TRUE}
    \end{itemize}

    \subsubsection{Nested Quantifiers}

    Suppose that \(\mathbb{N} = \{0,1,2,3,4,\dots\}\).
    \begin{itemize}
        \item \(\forall x \in \mathbb{N}, \exists y \in \mathbb{N}, x < y\) \(\rightarrow\) \emph{TRUE}
        \item \(\exists y \in \mathbb{N}, \forall x \in \mathbb{N}, x < y\) \(\rightarrow\) \emph{FALSE}
        \item \(\forall x \in \mathbb{N}, (x \geq 3 \text{ and } (\forall y \in \mathbb{N}, \forall z \in \mathbb{N}, (x = yz \Rightarrow y = 1 \text{ or } z = 1)) \Rightarrow \text{$x$ is odd})\)
    \end{itemize}

    \pagebreak

    \subsection{``Vacuously true''}

    \emph{Question:} Is the empty set a subset of every set? Yes. For \(A \subseteq B\) it means \(\forall x \in A, x \in B\). If $A$ is empty, this is \emph{vacuously} true.

    \vspace{1em}

    What does ``vacuously true'' mean? Suppose we have the following sets and statements:

    \begin{itemize}
        \item \(A = \{3,5,7,8,11\}\), \(\forall x \in A, P(x)\) \(\rightarrow\) \emph{FALSE}
        \item \(A = \{3,5,7\}\), \(\forall x \in A, P(x)\) \(\rightarrow\) \emph{TRUE}
        \item \(A = \{3\}\), \(\forall x \in A, P(x)\) \(\rightarrow\) \emph{TRUE}
        \item \(A = \emptyset\), \(\forall x \in A, P(x)\) \(\rightarrow\) \emph{``vacuously true''}
    \end{itemize}

    If $A$ is the empty set, the statement \(\forall x \in A, P(x)\) is \emph{always} true, no matter what \(P(x)\) is.

    \vspace{1em}

    Another example would be: ``All unicorns are green.'' This is true because there are 0 unicorns to check.
\end{document}